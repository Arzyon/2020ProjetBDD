% utf8

% teste pour fichier avec packages importés, mise en fomre comme exo7, ceci est un essai.




%%%%%%%%%%%%%%%%%%%%%%%%%%%%%%%%%%%%%%%%%%%%%%%%%%%%%%%%%%%%%%%%%%%%%%%%%%%%%%%%%%%%%%%%%%%%%%%%%%%%%%%%%%%%
%%%%%%%%%%%%%%%%%%CLASSE DU DOCUMENT%%%%%%%%%%%%%%%%%%%%%%%%%%%%%%%%%%%%%%%%%%%%%%%%%%%%%%%%%%%%%%%%%%%%%%%%
%%%%%%%%%%%%%%%%%%%%%%%%%%%%%%%%%%%%%%%%%%%%%%%%%%%%%%%%%%%%%%%%%%%%%%%%%%%%%%%%%%%%%%%%%%%%%%%%%%%%%%%%%%%%
\documentclass[12pt,a4paper]{article}


\usepackage{rennesson} %va chercher le fichier rennesson.sty


\begin{document}
%\pagecolor{couleur}%définir la couleur de fond de toute la page
%\pagecolor{noir_gris}
%{\color{vert_olive}

%titre, auteur et date en orange
%{\color{orange}
\title{Rendu intermédiaire N°1}%%replace X with the appropriate number
\author{BOUSTEÏLA Sarah 216248, RENNESSON Bastien 57119}%%replace with your name %%if necessary, replace with your course title
\maketitle
%}




%%%%%%%%%%%%%%%%%%%%%%%%%%%%%%%%%%%%%%%%%%%%%%%%%%%%%%%%%%%%%%%%%%%%%%%%%%%%%%%%%%%%%%%%%%%%%%%%%%%%%%%%%%%%
%%%%%%%%%%%%%%%%%%%%%%%%%%%%%%%%%%%%%%%%%%%%%%%%%%%%%%%%%%%%%%%%%%%%%%%%%%%%%%%%%%%%%%%%%%%%%%%%%%%%%%%%%%%%
%%%%%%%%%%%%%%%%%%%%%%%%%%%%%%%%%%%%%%%%%%%%%%%%%%%%%%%%%%%%%%%%%%%%%%%%%%%%%%%%%%%%%%%%%%%%%%%%%%%%%%%%%%%%
\section{Courte introduction sur les légendes utilisés dans le document.}
Dans l'ensemble du document, nous, désigne Sarah et Bastien; \\
le client, désigne Monsieur Chochoix.


\section{Schéma entité-association.}

ICI METTRE DESSIN SI POSSSIBLES




\subsection{Les hypothèses non trivialles.}
Puisque notre "schéma entité-association devra être accompagné d’un court texte justifiant les éventuels choix ou hypothèses que [nous avons] été amenés à faire", nous ne signalerons que les hypothèses non trivialles.

Les cardinalités minimale et maximale non expliquées, se basent sur le vocabulaire utilisé par le client dans le cahier des charges. Qui sont, pour rappel succint, "pour tous", "chaque", "peut", "parfois", "toujours" ... 



remplacer à la fin pour le propre
card min -> cardinalité minimum
card max -> cardinalité maximum
NNN -> Nous avons choisit 




NNN que la table spécialité comprend toutes les spécialités, tous domaines confondus, id est principale et secondaire. Ainsi : 
0N interprète, card min 0 car "parfois" possible qu'interprète n'ai pas de spécialité secondaire,
card maxi N car plusieurs
0N spécialité, card 0 min car toutes les spécialités secondaires ne sont pas présentes, certaines ne sont que des principales (table contient toutes spé, certaines ne sont que des ppal, donc pas prises pour secondaires), 
card N maxi car "ses spécialités secondaires". 


NNN 
NNN 
NNN 
NNN 
NNN 




LES CONCERTS
LES SALLES ET LEUR PERSONNEL
LA TARIFICATION DES CONCERTS



\subsection{Les compléments apportés directement par le client.}
\subsection{Les contraintes que votre modèle ne peut pas garantir.}


texte source à dez une fois les 3 parties du haut faites
Votre schéma entité-association devra être accompagné d’un court texte justifiant les éventuels choix ou hypothèses que vous avez été amenés à faire, les compléments apportés directement par le client, et les contraintes que votre modèle ne peut pas garantir et qui devront être vérifiées dans les parties suivantes.




%%%%%%%%%%%%%%%%%%%%%%%%%%%%%%%%%%%%%%%%%%%%%%%%%%%%%%%%%%%%%%%%%%%%%%%%%%%%%%%%%%%%%%%%%%%%%%%%%%%%%%%%%%%%
%%%%%%%%%%%%%%%%%%%%%%%%%%%%%%%%%%%%%%%%%%%%%%%%%%%%%%%%%%%%%%%%%%%%%%%%%%%%%%%%%%%%%%%%%%%%%%%%%%%%%%%%%%%%
%%%%%%%%%%%%%%%%%%%%%%%%%%%%%%%%%%%%%%%%%%%%%%%%%%%%%%%%%%%%%%%%%%%%%%%%%%%%%%%%%%%%%%%%%%%%%%%%%%%%%%%%%%%%



% \\
% \hspace*{2mm}
% \hspace*{2mm}=\hspace*{2mm}
% ${\dis{  }}$
% Soit ${\dis{ f \in \mathscr{F} \left(X, \K \right) }}$
% Soit ${\dis{  }}$
% {\color{red} blabla}
% {\color{orange} blabla}




}
\end{document}
%%%%%%%%%%%%%%%%%%%%%%%%%%%%%%%%%%%%%%%%%%%%%%%%%%%%%%%%%%%%%%%%%%%%%%%%%%%%%%%%%%%%%%%%%%%%%%%%%%%%%%%%%%%%
%%%%%%%%%%%%%%%%%%%%%%%%%%%%%%%%%%%%%%%%%%%%%%%%%%%%%%%%%%%%%%%%%%%%%%%%%%%%%%%%%%%%%%%%%%%%%%%%%%%%%%%%%%%%
%%%%%%%%%%%%%%%%%%%%%%%%%%%%%%%%%%%%%%%%%%%%%%%%%%%%%%%%%%%%%%%%%%%%%%%%%%%%%%%%%%%%%%%%%%%%%%%%%%%%%%%%%%%%
\section{}
\subsection{}
\subsubsection{}
%%%%%%%%%%%%%%%%%%%%%%%%%%%%%%%%%%%%%%%%%%%%%%%%%%%%%%%%%%%%%%%%%%%%%%%%%%%%%%%%%%%%%%%%%%%%%%%%%%%%%%%%%%%%
%%%%%%%%%%%%%%%%%%%%%%%%%%%%%%%%%%%%%%%%%%%%%%%%%%%%%%%%%%%%%%%%%%%%%%%%%%%%%%%%%%%%%%%%%%%%%%%%%%%%%%%%%%%%
%%%%%%%%%%%%%%%%%%%%%%%%%%%%%%%%%%%%%%%%%%%%%%%%%%%%%%%%%%%%%%%%%%%%%%%%%%%%%%%%%%%%%%%%%%%%%%%%%%%%%%%%%%%%
\section{}
\subsection{}
\subsubsection{}
%%%%%%%%%%%%%%%%%%%%%%%%%%%%%%%%%%%%%%%%%%%%%%%%%%%%%%%%%%%%%%%%%%%%%%%%%%%%%%%%%%%%%%%%%%%%%%%%%%%%%%%%%%%%
%%%%%%%%%%%%%%%%%%%%%%%%%%%%%%%%%%%%%%%%%%%%%%%%%%%%%%%%%%%%%%%%%%%%%%%%%%%%%%%%%%%%%%%%%%%%%%%%%%%%%%%%%%%%
%%%%%%%%%%%%%%%%%%%%%%%%%%%%%%%%%%%%%%%%%%%%%%%%%%%%%%%%%%%%%%%%%%%%%%%%%%%%%%%%%%%%%%%%%%%%%%%%%%%%%%%%%%%%
\section{}
\subsection{}
\subsubsection{}
%%%%%%%%%%%%%%%%%%%%%%%%%%%%%%%%%%%%%%%%%%%%%%%%%%%%%%%%%%%%%%%%%%%%%%%%%%%%%%%%%%%%%%%%%%%%%%%%%%%%%%%%%%%%
%%%%%%%%%%%%%%%%%%%%%%%%%%%%%%%%%%%%%%%%%%%%%%%%%%%%%%%%%%%%%%%%%%%%%%%%%%%%%%%%%%%%%%%%%%%%%%%%%%%%%%%%%%%%
%%%%%%%%%%%%%%%%%%%%%%%%%%%%%%%%%%%%%%%%%%%%%%%%%%%%%%%%%%%%%%%%%%%%%%%%%%%%%%%%%%%%%%%%%%%%%%%%%%%%%%%%%%%%
\section{}
%---------------------------------------------------------------
\subsection{}


\hspace*{2mm}
\hspace*{2mm}=\hspace*{2mm}
${\dis{  }}$
Soit ${\dis{ f \in \mathscr{F} \left(X, \K \right) }}$
Soit ${\dis{  }}$
{\color{red} blabla}
{\color{orange} blabla}

Soit ${\dis{E}}$ un $\K$-espace vectoriel {\color{orange} Attention! pas forcément en dimension finie} de dimension finie non nulle.

Soit ${\dis{n \in \N^{*} }}$ et soit Soit ${\dis{A \in \mathscr{M}_{n} \left( \K \right) }}$\\
 un $\K$-espace vectoriel {\color{orange} Attention! pas forcément en dimension finie} de dimension finie non nulle.
un $\K$-espace vectoriel {\color{orange} Attention! pas forcément en dimension finie} de dimension finie non nulle.
Soit ${\dis{A \in \mathscr{M}_{n} \left( \K \right) }}$
Soit $f$ un endomorphisme de $E$
${\dis{}}$


