% utf8

% teste pour fichier avec packages importés, mise en fomre comme exo7, ceci est un essai.




%%%%%%%%%%%%%%%%%%%%%%%%%%%%%%%%%%%%%%%%%%%%%%%%%%%%%%%%%%%%%%%%%%%%%%%%%%%%%%%%%%%%%%%%%%%%%%%%%%%%%%%%%%%%
%%%%%%%%%%%%%%%%%%CLASSE DU DOCUMENT%%%%%%%%%%%%%%%%%%%%%%%%%%%%%%%%%%%%%%%%%%%%%%%%%%%%%%%%%%%%%%%%%%%%%%%%
%%%%%%%%%%%%%%%%%%%%%%%%%%%%%%%%%%%%%%%%%%%%%%%%%%%%%%%%%%%%%%%%%%%%%%%%%%%%%%%%%%%%%%%%%%%%%%%%%%%%%%%%%%%%
\documentclass[12pt,a4paper]{article}


\usepackage{rennesson} %va chercher le fichier rennesson.sty


\begin{document}
%\pagecolor{couleur}%définir la couleur de fond de toute la page
%\pagecolor{noir_gris}
%{\color{vert_olive}

%titre, auteur et date en orange
%{\color{orange}
\title{Compte rendu}%%replace X with the appropriate number
\author{BOUSTEÏLA Sarah 216248, RENNESSON Bastien 57119}%%replace with your name %%if necessary, replace with your course title
\maketitle
%}




%%%%%%%%%%%%%%%%%%%%%%%%%%%%%%%%%%%%%%%%%%%%%%%%%%%%%%%%%%%%%%%%%%%%%%%%%%%%%%%%%%%%%%%%%%%%%%%%%%%%%%%%%%%%
%%%%%%%%%%%%%%%%%%%%%%%%%%%%%%%%%%%%%%%%%%%%%%%%%%%%%%%%%%%%%%%%%%%%%%%%%%%%%%%%%%%%%%%%%%%%%%%%%%%%%%%%%%%%
%%%%%%%%%%%%%%%%%%%%%%%%%%%%%%%%%%%%%%%%%%%%%%%%%%%%%%%%%%%%%%%%%%%%%%%%%%%%%%%%%%%%%%%%%%%%%%%%%%%%%%%%%%%%
\section{Courte introduction sur les légendes utilisés dans le document.}
Dans l'ensemble du document, nous, désigne Sarah et Bastien; le client, désigne Monsieur Chochoix.

\section{Présentation du projet et petit manuel utilisateur.}
\subsection{Présentation du projet.}
\subsection{Petit manuel utilisateur.}
\section{Modélisation et création de la base de données.}
\subsection{Schéma EA via le cahier des charges.}
\subsection{Du schéma EA au MLD.}
\section{Fonctionnalités du site web.}
\subsection{Structure du site web.}
\subsection{Requêtes SQL, leur fonctionnement et leur utilité.}
\section{Pistes d’améliorations du projet.}
\subsection{.}
\subsection{.}
\section{Organisation et la répartition du travail du binôme.}
Ce qui suit, nous ne savons pas s'il doit figurer au compte rendu du projet. Ou s'il a une quelconque valeur pédagogique ou non. Au cas où, c'est pourquoi il figure à la fin. 
Premièrement avant même d'avoir le sujet, nous avons convenu quelques règles, qui sont présentes sur le document Gestion_Projet_L2.pdf donné en pièce jointe. Bastien, ayant déjà validé son S3, a une charge de travail un peu réduite comparé à Sarah, ainsi il fait en avance -dans la mesure du possible- certains objectifs du travail en binôme. Attention, Bastien ne fait que sa partie du travail, le binôme est concient et équilibré sur la répartition des tâches et du travail fourni. Chaque membre du binôme effectue la tâche de travail qui lui est due, préalablement répartie équitablement entre Sarah et Bastien.
\subsection{Interprétation du sujet, lecture du cahier des charges et schéma EA.}
Nous, Sarah et Bastien, avons lu le cahier des charges séparément afin de faire notre propre schéma EA. Pour avoir un point de vue différent sur l'interprétation du cahier des charges. Dans le but de mettre en avant les forces et les faiblesses des différentes représentaions possibles, et de confronter nos mauvaises interprétations du sujet. 

\subsection{.}
\subsection{.}
\subsection{.}
\subsection{.}
\subsection{.}
\subsection{.}
\subsection{.}
\subsection{.}
\section{.}
\subsection{.}
\subsection{.}






\section{Lecture du sujet.}
\section{Déf CLEAN A TRIER }
\section{Déf CLEAN A TRIER }
\section{Déf CLEAN A TRIER }
\section{Remerciements}
\section{Bibliographie}
\section{Méthodoligie et répartion du travail du binôme.}
Ce qui suit, nous ne savons pas s'il doit figurer au compte rendu du projet. Ou s'il a une quelconque valeur pédagogique ou non. Au cas où, c'est pourquoi il figure à la fin. 
Premièrement avant même d'avoir le sujet, nous avons convenu quelques règles, qui sont présentes sur le document Gestion_Projet_L2.pdf donné en pièce jointe.

\ssubsection{Interprétation du sujet, lecture du cahier des charges et schéma EA.}
Nous, Sarah et Bastien, avons lu le cahier des charges séparément afin de faire notre propre schéma EA. Pour avoir un point de vue différent sur l'interprétation du cahier des charges. Dans le but de mettre en avant les forces et les faiblesses des différentes représentaions possibles, et de confronter nos mauvaises interprétations du sujet. 

\ssubsection{}
\ssubsection{}
\ssubsection{}



%%%%%%%%%%%%%%%%%%%%%%%%%%%%%%%%%%%%%%%%%%%%%%%%%%%%%%%%%%%%%%%%%%%%%%%%%%%%%%%%%%%%%%%%%%%%%%%%%%%%%%%%%%%%
%%%%%%%%%%%%%%%%%%%%%%%%%%%%%%%%%%%%%%%%%%%%%%%%%%%%%%%%%%%%%%%%%%%%%%%%%%%%%%%%%%%%%%%%%%%%%%%%%%%%%%%%%%%%
%%%%%%%%%%%%%%%%%%%%%%%%%%%%%%%%%%%%%%%%%%%%%%%%%%%%%%%%%%%%%%%%%%%%%%%%%%%%%%%%%%%%%%%%%%%%%%%%%%%%%%%%%%%%
\section{Propriétés des objets géométriques A PEAUFINER}



% \\
% \hspace*{2mm}
% \hspace*{2mm}=\hspace*{2mm}
% ${\dis{  }}$
% Soit ${\dis{ f \in \mathscr{F} \left(X, \K \right) }}$
% Soit ${\dis{  }}$
% {\color{red} blabla}
% {\color{orange} blabla}




}
\end{document}
%%%%%%%%%%%%%%%%%%%%%%%%%%%%%%%%%%%%%%%%%%%%%%%%%%%%%%%%%%%%%%%%%%%%%%%%%%%%%%%%%%%%%%%%%%%%%%%%%%%%%%%%%%%%
%%%%%%%%%%%%%%%%%%%%%%%%%%%%%%%%%%%%%%%%%%%%%%%%%%%%%%%%%%%%%%%%%%%%%%%%%%%%%%%%%%%%%%%%%%%%%%%%%%%%%%%%%%%%
%%%%%%%%%%%%%%%%%%%%%%%%%%%%%%%%%%%%%%%%%%%%%%%%%%%%%%%%%%%%%%%%%%%%%%%%%%%%%%%%%%%%%%%%%%%%%%%%%%%%%%%%%%%%
\section{}
\subsection{}
\subsubsection{}
%%%%%%%%%%%%%%%%%%%%%%%%%%%%%%%%%%%%%%%%%%%%%%%%%%%%%%%%%%%%%%%%%%%%%%%%%%%%%%%%%%%%%%%%%%%%%%%%%%%%%%%%%%%%
%%%%%%%%%%%%%%%%%%%%%%%%%%%%%%%%%%%%%%%%%%%%%%%%%%%%%%%%%%%%%%%%%%%%%%%%%%%%%%%%%%%%%%%%%%%%%%%%%%%%%%%%%%%%
%%%%%%%%%%%%%%%%%%%%%%%%%%%%%%%%%%%%%%%%%%%%%%%%%%%%%%%%%%%%%%%%%%%%%%%%%%%%%%%%%%%%%%%%%%%%%%%%%%%%%%%%%%%%
\section{}
\subsection{}
\subsubsection{}
%%%%%%%%%%%%%%%%%%%%%%%%%%%%%%%%%%%%%%%%%%%%%%%%%%%%%%%%%%%%%%%%%%%%%%%%%%%%%%%%%%%%%%%%%%%%%%%%%%%%%%%%%%%%
%%%%%%%%%%%%%%%%%%%%%%%%%%%%%%%%%%%%%%%%%%%%%%%%%%%%%%%%%%%%%%%%%%%%%%%%%%%%%%%%%%%%%%%%%%%%%%%%%%%%%%%%%%%%
%%%%%%%%%%%%%%%%%%%%%%%%%%%%%%%%%%%%%%%%%%%%%%%%%%%%%%%%%%%%%%%%%%%%%%%%%%%%%%%%%%%%%%%%%%%%%%%%%%%%%%%%%%%%
\section{}
\subsection{}
\subsubsection{}
%%%%%%%%%%%%%%%%%%%%%%%%%%%%%%%%%%%%%%%%%%%%%%%%%%%%%%%%%%%%%%%%%%%%%%%%%%%%%%%%%%%%%%%%%%%%%%%%%%%%%%%%%%%%
%%%%%%%%%%%%%%%%%%%%%%%%%%%%%%%%%%%%%%%%%%%%%%%%%%%%%%%%%%%%%%%%%%%%%%%%%%%%%%%%%%%%%%%%%%%%%%%%%%%%%%%%%%%%
%%%%%%%%%%%%%%%%%%%%%%%%%%%%%%%%%%%%%%%%%%%%%%%%%%%%%%%%%%%%%%%%%%%%%%%%%%%%%%%%%%%%%%%%%%%%%%%%%%%%%%%%%%%%
\section{}
%---------------------------------------------------------------
\subsection{}


\hspace*{2mm}
\hspace*{2mm}=\hspace*{2mm}
${\dis{  }}$
Soit ${\dis{ f \in \mathscr{F} \left(X, \K \right) }}$
Soit ${\dis{  }}$
{\color{red} blabla}
{\color{orange} blabla}

Soit ${\dis{E}}$ un $\K$-espace vectoriel {\color{orange} Attention! pas forcément en dimension finie} de dimension finie non nulle.

Soit ${\dis{n \in \N^{*} }}$ et soit Soit ${\dis{A \in \mathscr{M}_{n} \left( \K \right) }}$\\
 un $\K$-espace vectoriel {\color{orange} Attention! pas forcément en dimension finie} de dimension finie non nulle.
un $\K$-espace vectoriel {\color{orange} Attention! pas forcément en dimension finie} de dimension finie non nulle.
Soit ${\dis{A \in \mathscr{M}_{n} \left( \K \right) }}$
Soit $f$ un endomorphisme de $E$
${\dis{}}$


