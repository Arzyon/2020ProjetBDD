% utf8

% teste pour fichier avec packages importés, mise en fomre comme exo7, ceci est un essai.




%%%%%%%%%%%%%%%%%%%%%%%%%%%%%%%%%%%%%%%%%%%%%%%%%%%%%%%%%%%%%%%%%%%%%%%%%%%%%%%%%%%%%%%%%%%%%%%%%%%%%%%%%%%%
%%%%%%%%%%%%%%%%%%CLASSE DU DOCUMENT%%%%%%%%%%%%%%%%%%%%%%%%%%%%%%%%%%%%%%%%%%%%%%%%%%%%%%%%%%%%%%%%%%%%%%%%
%%%%%%%%%%%%%%%%%%%%%%%%%%%%%%%%%%%%%%%%%%%%%%%%%%%%%%%%%%%%%%%%%%%%%%%%%%%%%%%%%%%%%%%%%%%%%%%%%%%%%%%%%%%%
\documentclass[12pt,a4paper]{article}


\usepackage{rennesson} %va chercher le fichier rennesson.sty


\begin{document}
%\pagecolor{couleur}%définir la couleur de fond de toute la page
\pagecolor{noir_gris}
{\color{vert_olive}

%titre, auteur et date en orange
{\color{orange}
\title{JOURNAL}%%replace X with the appropriate number
\author{BOUSTEÏLA Sarah 216248, RENNESSON Bastien 57119}%%replace with your name %%if necessary, replace with your course title
\maketitle
}




%%%%%%%%%%%%%%%%%%%%%%%%%%%%%%%%%%%%%%%%%%%%%%%%%%%%%%%%%%%%%%%%%%%%%%%%%%%%%%%%%%%%%%%%%%%%%%%%%%%%%%%%%%%%
%%%%%%%%%%%%%%%%%%%%%%%%%%%%%%%%%%%%%%%%%%%%%%%%%%%%%%%%%%%%%%%%%%%%%%%%%%%%%%%%%%%%%%%%%%%%%%%%%%%%%%%%%%%%
%%%%%%%%%%%%%%%%%%%%%%%%%%%%%%%%%%%%%%%%%%%%%%%%%%%%%%%%%%%%%%%%%%%%%%%%%%%%%%%%%%%%%%%%%%%%%%%%%%%%%%%%%%%%
\section{Lecture du sujet et construction du schéma EA.}
\subsection{Les concerts}
2020_10_08 1400 BU ETAGE 2, 

première contrainte du sujet :"Un organisme gère l'organisation d'un festival de musique classique dans une région de la France sur une période de 3 semaines."
Donc 
type de musique = musique classique
lieu terreste = une région de la France (regarder si dep outre-mer et région outre-mer comptent)
durée = 3 semaines (mettre verif dans date des évènements)


Info complémentaire pour BDD à 1ère lecture
De chaque interprète, on connaît nom, prénom, téléphone, email, date de naissance, nationalité et résumé de sa biographie. ==> NOT NULL pour table interprete







Rappel de Cho (changer par Monsieur Chochoix pour toute la suite à la fin au propre)
Éviter dans association être et avoir.
Recherche de tous les synonimes de être et avoir

https://www.synonymes.com/synonyme.php?mot=avoir
avoir : 
appartenance, posséder, disposer de, bénéficier de, être propriétaire de, jouir de, détenir, commerce, acquérir, acheter, se procurer, fait, obtenir, gagner, recevoir, remporter, toucher, état, vêtement, porter, émotion, éprouver, sentir, ressentir, relation, tromper, rouler, attraper, disposer, garder, renfermer, tenir

https://www.synonymes.com/synonyme.php?mot=%EAtre
être : 
exister, subsister, vivre, réaliser, avoir lieu, se faire, avoir été, se montrer, se présenter, se trouver, se voir, aller, représenter, appartenir, accomplir, résider, tenir, trouver



















Essaye de trouver les entités et associations, relecture multiples et aide pour le schéma EA sur mocodo



#CARDINALITÉ
Analyse syntaxique du français, i.e sur les possibilités des cardinalités
0-1     aucune ou une seule
1-1     une et une seule
0-N     aucune ou plusieurs
1-N     une ou plusieurs


À partir des phrases du sujet
"Les concerts ont parfois un thème" ==> thématique, 01 concert, 1N thème
d'ou 01 concert,
1N thème, car organisme ne garde que les thèmes utilisés sur les 3 semaines


@"La plupart des concerts ont un entracte et on sait alors quand il se situe."
==> bénéficier de, 01 concert, 11 entracte
01 concert car certain concert n'ont pas d'entracte
11 entracte car entracte id = 4 est uniquement utilisée pour le concert id = 7


"A chaque concert sont programmés, dans un ordre précis; plusieurs morceaux. "
==> programme, 1N morceau, 1N concert: ordre (position)
1N morceau car il y a entre 1 et N morceaux dans un concert
1N concert car 1 morceau peut être joué dans plusieurs concerts, 1 card min car organisme sait exactement
tous les morceaux qui seront joués en concert. Il n'y a aucun intérêt à avoir les morceaux non joués.
RMQ ordre doit être unique, mais on ne sait pas (juste section Les concerts) si un morceu est joué plusieurs fois dans un concert (pour les contraintes)



2@ "Chaque morceau peut avoir plusieurs interprètes."
==> interpréter, 1N morceau, XX interprète
1N morceau card mini 1 pour soliste, N pour orchestre
Au premier abord on mettrait 1N interprète sauf que
"Seuls les principaux interprètes sont mentionnés (et non pas tous les musiciens)."
Ainsi on ne veut garder que les plus importants, ce qui donne
01 interprète, card mini 0 signifie que l'interprète n'est pas interprète principal, 1 il l'est
D'où ==> interpréter, 1N morceau, 01 interprète



3@ "Il arrive que le même morceau soit joué dans le même concert par des interprètes différents."
JE NE SAIS PAS COMMENT L'INTERPRÉTER POUR L'INSTANT, PE DEMANDE UNE NOUVELLE RELATION
EN ATTENTE



"A chaque concert correspond une affiche"
==> correspond, 11 concert, 1N affiche
11 concert, on est sur une région de France, ainsi une même affiche pour toute la FRANCE, ce qui serait différent si on travaillait sur le monde entier. Merci Cho
1N affiche, car les dates y sont imprimées dessus, concert 1 jour ou sur 3 semaines de représentations


"Certains concerts disposent de plus d'une bande sonore promotionnelle sous forme de fichier."
==> disposer, 01 concert, 11 bande_promo
01 concert, car "Certains concerts disposent", d'où card mini 0, card maxi 1 car une seule bande promo pour le concert, qui est différent du cinéma qui montre jusqu'à 3 bandes annonces différentes.
11 bande_promo, card mini 1 une bande promo qui n'a pas de concert est un non sens, card maxi lui est beaucoup plus ambigüe, 
	-cas 1: une unique répresentation; donne 1
	-cas 2: des représentations sur 3 semaines; donne n
Sauf que dans sujet "Les concerts se déroulent, à diverses dates et heures, dans différentes salles de spectacle." Ainsi comme pour https://www.sortiraparis.com/scenes/concert-musique/articles/209494-saint-seiya-symphonic-adventure-les-chevaliers-du-zodiaque-en-concert-symphoniqu
nous sommes dans le cas 2. D'où 
==> disposer, 01 concert, 1n bande_promo



" Sa spécialité principale est mentionnée (violon, harpe...) ainsi que, parfois, ses spécialités secondaires."
==> secondaire, 0N interprète, 0N spécialité
0N interprète, card min car 0 car "parfois" possible qu'interprète n'ai pas de spécialité secondaire,
card maxi N car plusieurs
0N spécialité, card 0 min car toutes les spécialités secondaires ne sont pas présentes, certaines ne sont que dess principales (table contient toutes spé, certaines ne sont que des ppal, donc pas prises pour secondaires), 
card N maxi car "ses spécialités secondaires".




dernier plan en cours
#######################MOCODO_MOCODO_MOCODO_MOCODO_MOCODO###################################
secondaire, 0N interprète, 0N spécialité
spécialité: id_spécialité, nom (ex: violon, harpe)
theme: id_theme, nom, XXX
entracte: id_entracte, nom, situe (=position en temps ou après X morceaux), XXX

interprète: id_interprète, nom, prénom, tel, email, naiss, nationalité, résumé_bio,
principale, 11 interprète, 0N spécialité
thématique, 01 concert, 1N theme
bénéficier de, 01 concert, 11 entracte

interpréter, 1N morceau, 01 interprète
programme, 1N morceau, 1N concert: ordre (position)
concert: id_concert, nom, date, heure, salle_de_spectacle, XXX
enregistrer, 01 concert, 11 disque

morceau: id_morceau, auteur, titre, durée
affiche: id_affiche, img_XL_impression, img_com_web, colorimétrie, profondeur_couleur
correspond, 11 concert, 1N affiche
disque: id_disque, nom, prix_disque

:
:
bande_promo: id_bande_promo, taille (en ko), format (ex: MIME), mots clés, durée 
disposer, 01 concert, 1n bande_promo

#############################################################################################







Liste des questions à posser à Cho
1 sur @ : "La plupart des concerts ont un entracte et on sait alors quand il se situe."
Remarque d' Alexis sur discord
perso j'aurais mis 0,n pour la cardinalité des entractes. car même si dans le sujet il disent que les concert peuvent avoir un entracte, en vrai si c'est un opera il peut y a voir plusieurs entracte (mais ça c'est chipoter, et puis si c'est dans le sujet ils font ce qu'ils veulent)

Q : Dans vos opéras, il y a bien une seule entracte ?
SI oui, rien ne change
Sinon, ==>  bénéficier de, 01 concert, 1N entracte



2 sur 2@ : "Chaque morceau peut avoir plusieurs interprètes."
==> interpréter, 1N morceau, XX interprète
1N morceau card mini 1 pour soliste, N pour orchestre
Au premier abord on mettrait 1N interprète sauf que
"Seuls les principaux interprètes sont mentionnés (et non pas tous les musiciens)."
Ainsi on ne veut garder que les plus importants, ce qui donne
01 interprète, card mini 0 signifie que l'interprète n'est pas interprète principal, 1 il l'est
D'où ==> interpréter, 1N morceau, 01 interprète

Q: Mon interprétation de la phrase et des cardinalité est-elle correcte ?
SI oui, rien ne change
Sinon, faire les modifications selon la réponse de Cho



3 sur 3@ : "Il arrive que le même morceau soit joué dans le même concert par des interprètes différents."
JE NE SAIS PAS COMMENT L'INTERPRÉTER POUR L'INSTANT, PE DEMANDE UNE NOUVELLE RELATION
EN ATTENTE

Q: Vous voulez dire que l'interprétation de la marche de l'empereur de Stars war joué en premier
dans le concert par l'interprète Philippe Cho, puis par Sarah BOUSTEÏLA ?s

FAIRE LES MODIFICATIONS SELON LA RÉPONSE DE CHO




%%%%%%%%%%%%%%%%%%%%%%%%%%%%%%%%%%%%%%%%%%%%%%%%%%%%%%%%%%%%%%%%%%%%%%%%%%%%%%%%%%%%%%%%%%%%%%%%%%%%%%%%%%%%
\subsection{Les salles et leur personnel}
2020_10_09 0800 SALLE 2B100


dernier plan en cours
#######################MOCODO_MOCODO_MOCODO_MOCODO_MOCODO###################################

salle: nom(varchar(42)), adr, directeur (=id_personne), chargé de communication (=id_personne), régisseur en chef(=id_personne)
référencé, 11 salle, 1N lieu
:

travailleur, 0N personne, 1N salle
lieu: id_lieu, nom, adr, localité, dep, soustype (NULL),
se dérouler, 1N concert, 1N lieu

personne: id_personne, nom, prénom, téléphone, email
souslieu, 01 lieu,   1N> [lieu] lieu: id_lieu
concert: id_concert, nom, date, heure, salle_de_spectacle, XXX

#############################################################################################



À partir des phrases du sujet

"Les concerts se déroulent dans différents lieux."
==> se dérouler, 1N concert, 1N lieu
1N concert, On se base sur des concerts qui vont avoir lieu, donc pas de 0 en card mini
Si concert annulé, alors, on met info ailleurs
1N lieu, card 1 mini car lieu utilisé au moins 1 fois (hyp avant annulation),
car N maxi car un lieu accueil est utilisé pour plusieurs concerts 


"Pour chaque salle, on connaît ... son régisseur en chef."
régisseur en chef, 01 personne, 11 salle
01 personne, card mini 0 car n'est pas forcément régisseur en chef
car maxi 1, car si oui l'est une fois pour une salle donnée
11 salle, car mini 1 car une salle doit avoir 1 régisseur en chef et un seul

idem que ci-dessus
"Pour chaque salle, on connaît ... son directeur"
"Pour chaque salle, on connaît ... son chargé de communication"



référencé, 11 salle, 1N lieu
1 salle est reférencée que dans un seul lieu
1 lieu à entre 1 et N salles



Info complémentaire pour BDD à 1ère lecture
"On possède sur chaque personne concernée une fiche avec son nom, prénom, téléphone, email."
==> implique NOT NULL pour la table concernée





%%%%%%%%%%%%%%%%%%%%%%%%%%%%%%%%%%%%%%%%%%%%%%%%%%%%%%%%%%%%%%%%%%%%%%%%%%%%%%%%%%%%%%%%%%%%%%%%%%%%%%%%%%%%
\subsection{Actualisation MOCODO schéma EA}
Juste la fusion des infos des sections précèdentes.

fusion des derniers plans en cours
#######################MOCODO_MOCODO_MOCODO_MOCODO_MOCODO###################################

affiche: id_affiche, img_XL_impression, img_com_web, colorimétrie, profondeur_couleur
disque: id_disque, nom, prix_disque
theme: id_theme, nom, XXX
spécialité: id_spécialité, nom (ex: violon, harpe)
principale, 11 interprète, 0N spécialité

correspond, 11 concert, 1N affiche
enregistrer, 01 concert, 11 disque
thématique, 01 concert, 1N theme
secondaire, 0N interprète, 0N spécialité
interprète: id_interprète, nom, prénom, tel, email, naiss, nationalité, résumé_bio,

bénéficier de, 01 concert, 11 entracte
concert: id_concert, nom, date, heure, salle_de_spectacle, XXX
programme, 1N morceau, 1N concert: ordre (position)
morceau: id_morceau, auteur, titre, durée
interpréter, 1N morceau, 01 interprète

entracte: id_entracte, nom, situe (=position en temps ou après X morceaux), XXX
se dérouler, 1N concert, 1N lieu
disposer, 01 concert, 1n bande_promo
bande_promo: id_bande_promo, taille (en ko), format (ex: MIME), mots clés, durée
personne: id_personne, nom, prénom, téléphone, email

souslieu, 01 lieu,   1N> [lieu] lieu: id_lieu
lieu: id_lieu, nom, adr, localité, dep, soustype (NULL),
référencé, 11 salle, 1N lieu
salle: nom(varchar(42)), adr, directeur (=id_personne), chargé de communication (=id_personne), régisseur en chef(=id_personne)
travailleur, 0N personne, 1N salle

#############################################################################################v





%%%%%%%%%%%%%%%%%%%%%%%%%%%%%%%%%%%%%%%%%%%%%%%%%%%%%%%%%%%%%%%%%%%%%%%%%%%%%%%%%%%%%%%%%%%%%%%%%%%%%%%%%%%%
\subsection{La tarification des concerts}


"Le prix ne dépend pas de la salle, ni du concert, mais de la catégorie de la place."

"Les places sont réparties en 3 catégories. On aura les places de catégorie 1 (excellente) à 50 €, de catégorie 2 (normale) à 30 €, et de catégorie 3 (visibilité réduite) à 15 €."

catégorie
id_catégorie, nom, prix
1,excellente, 50
2,normale, 30
3,visibilité réduite, 15



4@ appartenir, 11 place, 1N catégorie
11 place, card mini 1, card maxi 1 car une place de plusieurs catégorie n'existe pas.
1N catégorie, card mini 1 car, card maxi 1, 

Q: y a-t-il des concerts avec un tarif unique (mais calqué sur les catégories existantes)?




"Par exemple, le cirque d'hiver d'Amiens (code postal 80000) comporte 300 places de catégorie 1, aucune de catégorie 2 et 100 de catégorie 3."
==> dit que comporter(ticket), 1N lieu, 1N place, 0N catégorie: nb_place, prix_final(selon cas)
1N lieu, card mini 1 car lieu possède au moins une place, card maxi N car plusieurs places (siège ds la salle)
1N place, card mini 1 car propose au moins une place, card maxi N car tous les sièges.
0N catégorie, card mini 0 car ne propose toutes les catégories de places à  que tous les tarifs sont proposés, card maxi car

==> finalement 
==> comporter(ticket), 1N lieu, 1N place, 0N catégorie: prix_final(selon cas)



"Le prix ne dépend pas de la salle, ni du concert, mais de la catégorie de la place. Les places sont réparties en 3 catégories. On aura les places de catégorie 1 (excellente) à 50 €, de catégorie 2 (normale) à 30 €, et de catégorie 3 (visibilité réduite) à 15 €. Par exemple, le cirque d'hiver d'Amiens (code postal 80000) comporte 300 places de catégorie 1, aucune de catégorie 2 et 100 de catégorie 3.
Certaines villes ajoutent une taxe sur le tarif de base. Ainsi une place en catégorie 1 à Paris ne sera pas 50€ mais 60€. D’autres villes, au contraire, subventionnent les tarifs de base. Ainsi, une place en catégorie 2 à Dunkerque ne coutera pas 30€ mais 25€."
==> faire requete SQL pour ajuster sselon les cas prix de la place, qui dépend de politique du LIEU



dernier plan en cours
#######################MOCODO_MOCODO_MOCODO_MOCODO_MOCODO###################################


lieu: id_lieu, nom, adr, localité, dep, soustype (NULL),
comporter(ticket), 1N lieu, 1N place, 0N catégorie: prix_final(selon cas)
place: id_place, XXXstatut(vendu/réservé FNAC)

:
catégorie: id_catégorie, nom, prix_ref

#############################################################################################





%%%%%%%%%%%%%%%%%%%%%%%%%%%%%%%%%%%%%%%%%%%%%%%%%%%%%%%%%%%%%%%%%%%%%%%%%%%%%%%%%%%%%%%%%%%%%%%%%%%%%%%%%%%%
\subsection{Actualisation MOCODO schéma EA 2}
Juste la fusion des infos des sections précèdentes.

fusion des derniers plans en cours
#######################MOCODO_MOCODO_MOCODO_MOCODO_MOCODO###################################

affiche: id_affiche, img_XL_impression, img_com_web, colorimétrie, profondeur_couleur
disque: id_disque, nom, prix_disque
theme: id_theme, nom, XXX
spécialité: id_spécialité, nom (ex: violon, harpe)
principale, 11 interprète, 0N spécialité

correspond, 11 concert, 1N affiche
enregistrer, 01 concert, 11 disque
thématique, 01 concert, 1N theme
secondaire, 0N interprète, 0N spécialité
interprète: id_interprète, nom, prénom, tel, email, naiss, nationalité, résumé_bio,

bénéficier de, 01 concert, 11 entracte
concert: id_concert, nom(???), date, heure, salle_de_spectacle, XXX
programme, 1N morceau, 1N concert: ordre (position)
morceau: id_morceau, auteur, titre, durée
interpréter, 1N morceau, 01 interprète

entracte: id_entracte, nom, situe (=position en temps ou après X morceaux), XXX
se dérouler, 1N concert, 1N lieu
disposer, 01 concert, 1n bande_promo
bande_promo: id_bande_promo, taille (en ko), format (ex: MIME), mots clés, durée
personne: id_personne, nom, prénom, téléphone, email

souslieu, 01 lieu,   1N> [lieu] lieu: id_lieu
lieu: id_lieu, nom, adr, localité, dep, soustype (NULL),
référencé, 11 salle, 1N lieu
salle: nom(varchar(42)), adr, directeur (=id_personne), chargé de communication (=id_personne), régisseur en chef(=id_personne)
travailleur, 0N personne, 1N salle

catégorie: id_catégorie, nom, prix_ref
comporter(ticket), 1N lieu, 1N place, 0N catégorie: prix_final(selon cas)
place: id_place, XXXstatut(vendu/réservé FNAC)

#############################################################################################




%%%%%%%%%%%%%%%%%%%%%%%%%%%%%%%%%%%%%%%%%%%%%%%%%%%%%%%%%%%%%%%%%%%%%%%%%%%%%%%%%%%%%%%%%%%%%%%%%%%%%%%%%%%%
\subsection{Note complémentaire pour le jeu d'essai}


"La ville d'Amiens comporte 3 salles de spectacles pour ce festival."
==> Il faut 3 entrées dans table lieu avec, localité = Amiens



"Le département 80 comprend uniquement 2 villes participant au festival : Amiens et Abbeville."
==> 
SELECT DISTINCT lieu.localite
FROM lieu
WHERE lieu.localite = 80;

doit donner 2 en réponse. Amiens et Abbeville



"Le concerto pour clarinette de Mozart est interprété plusieurs fois au même concert du 15 janvier au Zénith d'Amiens par des interprètes différents."
==> plusieurs représentations 
table concert(id_concert, nom, date, heure, salle_de_spectacle, XXX)
INSERT INTO concert(id_concert, nom, date, heure, salle_de_spectacle) VALUES (1, 'Le concerto pour clarinette de Mozart', '15-01-2021', 'XXXX', 'Zénith d'Amiens');

"est interprété plusieurs fois au même concert du 15 janvier"



select id_concert
from concert
where concert.titre = 'Le concerto pour clarinette' and XXXXXXa continuer



- L'adagio d'Albinoni est interprété au concert du 16 janvier à 20H30 au palais de sports d'Amiens.





%%%%%%%%%%%%%%%%%%%%%%%%%%%%%%%%%%%%%%%%%%%%%%%%%%%%%%%%%%%%%%%%%%%%%%%%%%%%%%%%%%%%%%%%%%%%%%%%%%%%%%%%%%%%
\subsection{Site web}


info TD 2
Bien regarder si on peut revenir à l'état précedent pour savoir si choix entre binaire et ternaire.


exo
cf photo, explication table DateAffectation est supprimée car info est dans Affectation.





%%%%%%%%%%%%%%%%%%%%%%%%%%%%%%%%%%%%%%%%%%%%%%%%%%%%%%%%%%%%%%%%%%%%%%%%%%%%%%%%%%%%%%%%%%%%%%%%%%%%%%%%%%%%
%%%%%%%%%%%%%%%%%%%%%%%%%%%%%%%%%%%%%%%%%%%%%%%%%%%%%%%%%%%%%%%%%%%%%%%%%%%%%%%%%%%%%%%%%%%%%%%%%%%%%%%%%%%%
%%%%%%%%%%%%%%%%%%%%%%%%%%%%%%%%%%%%%%%%%%%%%%%%%%%%%%%%%%%%%%%%%%%%%%%%%%%%%%%%%%%%%%%%%%%%%%%%%%%%%%%%%%%%
\section{Déf CLEAN A TRIER }
\section{Déf CLEAN A TRIER }
\section{Déf CLEAN A TRIER }
\section{Déf CLEAN A TRIER }




%%%%%%%%%%%%%%%%%%%%%%%%%%%%%%%%%%%%%%%%%%%%%%%%%%%%%%%%%%%%%%%%%%%%%%%%%%%%%%%%%%%%%%%%%%%%%%%%%%%%%%%%%%%%
%%%%%%%%%%%%%%%%%%%%%%%%%%%%%%%%%%%%%%%%%%%%%%%%%%%%%%%%%%%%%%%%%%%%%%%%%%%%%%%%%%%%%%%%%%%%%%%%%%%%%%%%%%%%
%%%%%%%%%%%%%%%%%%%%%%%%%%%%%%%%%%%%%%%%%%%%%%%%%%%%%%%%%%%%%%%%%%%%%%%%%%%%%%%%%%%%%%%%%%%%%%%%%%%%%%%%%%%%
\section{Propriétés des objets géométriques A PEAUFINER}



% \\
% \hspace*{2mm}
% \hspace*{2mm}=\hspace*{2mm}
% ${\dis{  }}$
% Soit ${\dis{ f \in \mathscr{F} \left(X, \K \right) }}$
% Soit ${\dis{  }}$
% {\color{red} blabla}
% {\color{orange} blabla}




}
\end{document}
%%%%%%%%%%%%%%%%%%%%%%%%%%%%%%%%%%%%%%%%%%%%%%%%%%%%%%%%%%%%%%%%%%%%%%%%%%%%%%%%%%%%%%%%%%%%%%%%%%%%%%%%%%%%
%%%%%%%%%%%%%%%%%%%%%%%%%%%%%%%%%%%%%%%%%%%%%%%%%%%%%%%%%%%%%%%%%%%%%%%%%%%%%%%%%%%%%%%%%%%%%%%%%%%%%%%%%%%%
%%%%%%%%%%%%%%%%%%%%%%%%%%%%%%%%%%%%%%%%%%%%%%%%%%%%%%%%%%%%%%%%%%%%%%%%%%%%%%%%%%%%%%%%%%%%%%%%%%%%%%%%%%%%
\section{}
\subsection{}
\subsubsection{}
%%%%%%%%%%%%%%%%%%%%%%%%%%%%%%%%%%%%%%%%%%%%%%%%%%%%%%%%%%%%%%%%%%%%%%%%%%%%%%%%%%%%%%%%%%%%%%%%%%%%%%%%%%%%
%%%%%%%%%%%%%%%%%%%%%%%%%%%%%%%%%%%%%%%%%%%%%%%%%%%%%%%%%%%%%%%%%%%%%%%%%%%%%%%%%%%%%%%%%%%%%%%%%%%%%%%%%%%%
%%%%%%%%%%%%%%%%%%%%%%%%%%%%%%%%%%%%%%%%%%%%%%%%%%%%%%%%%%%%%%%%%%%%%%%%%%%%%%%%%%%%%%%%%%%%%%%%%%%%%%%%%%%%
\section{}
\subsection{}
\subsubsection{}
%%%%%%%%%%%%%%%%%%%%%%%%%%%%%%%%%%%%%%%%%%%%%%%%%%%%%%%%%%%%%%%%%%%%%%%%%%%%%%%%%%%%%%%%%%%%%%%%%%%%%%%%%%%%
%%%%%%%%%%%%%%%%%%%%%%%%%%%%%%%%%%%%%%%%%%%%%%%%%%%%%%%%%%%%%%%%%%%%%%%%%%%%%%%%%%%%%%%%%%%%%%%%%%%%%%%%%%%%
%%%%%%%%%%%%%%%%%%%%%%%%%%%%%%%%%%%%%%%%%%%%%%%%%%%%%%%%%%%%%%%%%%%%%%%%%%%%%%%%%%%%%%%%%%%%%%%%%%%%%%%%%%%%
\section{}
\subsection{}
\subsubsection{}
%%%%%%%%%%%%%%%%%%%%%%%%%%%%%%%%%%%%%%%%%%%%%%%%%%%%%%%%%%%%%%%%%%%%%%%%%%%%%%%%%%%%%%%%%%%%%%%%%%%%%%%%%%%%
%%%%%%%%%%%%%%%%%%%%%%%%%%%%%%%%%%%%%%%%%%%%%%%%%%%%%%%%%%%%%%%%%%%%%%%%%%%%%%%%%%%%%%%%%%%%%%%%%%%%%%%%%%%%
%%%%%%%%%%%%%%%%%%%%%%%%%%%%%%%%%%%%%%%%%%%%%%%%%%%%%%%%%%%%%%%%%%%%%%%%%%%%%%%%%%%%%%%%%%%%%%%%%%%%%%%%%%%%
\section{}
%---------------------------------------------------------------
\subsection{}


\hspace*{2mm}
\hspace*{2mm}=\hspace*{2mm}
${\dis{  }}$
Soit ${\dis{ f \in \mathscr{F} \left(X, \K \right) }}$
Soit ${\dis{  }}$
{\color{red} blabla}
{\color{orange} blabla}

Soit ${\dis{E}}$ un $\K$-espace vectoriel {\color{orange} Attention! pas forcément en dimension finie} de dimension finie non nulle.

Soit ${\dis{n \in \N^{*} }}$ et soit Soit ${\dis{A \in \mathscr{M}_{n} \left( \K \right) }}$\\
 un $\K$-espace vectoriel {\color{orange} Attention! pas forcément en dimension finie} de dimension finie non nulle.
un $\K$-espace vectoriel {\color{orange} Attention! pas forcément en dimension finie} de dimension finie non nulle.
Soit ${\dis{A \in \mathscr{M}_{n} \left( \K \right) }}$
Soit $f$ un endomorphisme de $E$
${\dis{}}$


